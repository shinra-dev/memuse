\section{Using the memuse Package}

The following sections demonstrate the ways in which the user is meant to interact with the \pkg{memuse} package.  Hence the title.


\subsection{Constructing memuse Objects}

The \code{memuse} object is an S4 class object, which is a high-falootin way of saying it's a data structure with specialized interpreted context.  Think of it like a list whose elements are always the same.  The specification is:

\begin{align*}
  \text{memuse} &= 
  \begin{cases}
    \text{size}\\
    \text{unit}\\
    \text{unit.prefix}\\
    \text{unit.names}
  \end{cases}
\end{align*}

This object has a prototype, sanity checking, and all kinds of other boring crap no one cares about.  What's important is how to use this thing.

To construct a \code{memuse} object, you can use the \code{memuse()} or \code{mu()} constructors.  These functions behave identically; \code{memuse()} exists because I generally find it inappropriate to not have an object constructor of the same name as the object, and \code{mu()} exists because if I have to type more than 5 characters, I'm completely furious.  I'm looking at you, \code{suppressPackageStartupMessages()}\dots

Precedence is given to \code{prefix} values over \code{unit} in the constructor.  So for example, \code{mu(10, "mb", prefix="IEC")} will return \code{10.000 MiB}.  The assumption is that you either do not know or do not care about the distinction between IEC and SI unit prefices, but are probably more familiar with IEC.



\subsection{Methods}
Aside from the constructor, you have already seen one very useful method:  \code{swap.prefix()}.  In addition to these, we have several other obvious methods, such as \code{swap.unit()}, \code{swap.names()}, \code{print()}, \code{show()}, etc.  But we also have some simple arithmetic, namely \code{`+`} (addition), \code{`*`} (multiplication), and \code{`\^{}`} (exponentiation).  So for example:
\begin{lstlisting}[language=rr]
mu(100) + mu(200)
## 300.000 B
mu(100) * mu(200) # 100*200/1024
## 19.531 KiB
\end{lstlisting}
Other arithmetic of memuse objects is available, including division, as well as 

Finally, we have the methods that inspired the creation of this entire dumb thing in the first place:  \code{howbig()} and \code{howmany()}.  The former takes in the dimensions of a matrix (\code{nrow} rows and \code{ncol} columns) and returns the memory usage (as the package namesake would imply) of the object.  So for example, if you wanted to perform a principal components decomposition on a 100,000 by 100,000 matrix via SVD (as we have), then you would need:
\begin{lstlisting}
howbig(100000, 100000)
## 74.506 GiB
\end{lstlisting}
Of ram just to store the data.  Another interesting anecdote about this sized matrix is that we were able to generate it in just over a tenth of a second.  Pretty cool, eh?

As mentioned before, there is also the \code{howmany()} method which does somewhat the reverse of \code{howbig()}.  Here you pass a \code{memuse} object and get a matrix size out.  You can pass (exactly) one argument \code{nrow} or \code{ncol} in addition to the \code{memuse} object; the method will determine the maximum possible size of the outlying dimension in the obvious way.  If no additional argument is passed, then the largest square matrix dimensions will be returned.



\subsection{Package Demos}

In addition to all of the above, the \pkg{memuse} package includes several demos.  You can execute them via the command:
\begin{lstlisting}[title=List of Demos]
### (Use Rscript.exe for windows systems)

# Basic construction/use of memuse objects
Rscript -e "demo(demo, package='memuse', ask=F, echo=F)"
# Arithmetic
Rscript -e "demo(demo2, package='memuse', ask=F, echo=F)"
# howbig/howmany examples
Rscript -e "demo(demo3, package='memuse', ask=F, echo=F)"
\end{lstlisting}
